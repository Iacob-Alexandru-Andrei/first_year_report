% \listoftodos[Background]
\todo{Scalability experiments with increasing cohort sizes}

The standard FL objective can be modelled as seen in \cref{eq:flObjective}
\begin{equation} \label{eq:flObjective}
    \underset{\theta}{\min} F(\theta) = \sum_{c \in C} p_c F_c(\theta) \ ,
\end{equation}
where \(F\) is the federated objective, $C$ is the client set, $\theta$ is the model, and \(F_c\) is the loss of client \(c\) weighted by their fraction of the total number of examples $p_c$. This formulation assumes that a single global model is being trained without regard for the distribution of its performance across client datasets. Federated Averaging (FedAvg)~\citep{FedAvg} trains the global model locally, for each round $t$ it sums the update \(\theta_t^c - \theta_t\) from client $c$ weighted by \(p_c\) with the previous model \(\theta_t\) using learning rate \( \eta \), as seen in \cref{eq:FedAvg}
\begin{equation} \label{eq:FedAvg}
    \theta_{t+1} = \theta_t + \eta \left( \sum_{c \in C} p_c \left(\theta_t^c - \theta_t \right) \ \right) \ .
\end{equation}

The inability to colocate client data and the need to construct rough mixtures of model parameters as a compromise represent the leading causes of FL-specific challenges.

\section{Heterogeneity}\label{background:data_heterogeneity}

Non-IID data has been shown to impact both practical accuracies \citep{FLwithNonIID, NonIIDQagmire} and theoretical convergence bounds \citep{OnTheConvergenceOfFedAvgOnNonIIDdata}. It is thus worth detailing some forms of heterogeneity that \citet{AdvancedAndOpenProblems} identify. The most commonly addressed form is quantity skew caused by clients having different amounts of data available. Standard FL algorithms effectively address Quantity skew via a simple reweighing (\cref{eq:FedAvg}). The other frequently-considered type of heterogeneity is label-distribution skew which is quantity skew per class. While these forms of heterogeneity have been most investigated, situations where features and labels are not related in the same manner across clients are far more pathological and may require some form of clustering or personalisation to tackle. In the worst-case scenario, each client may represent an entirely different task, as in Multi-Task Learning, with potentially no overlap in their solution space.


\paragraph{System (hardware) heterogeneity} Devices within the federated network may differ regarding computational ability, storage, network speed, and reliability. They may also differ from themselves at a different point in time as their battery power, network connection, or operational mode vary. Importantly, variations in data-generating hardware, such as sensors, are linked to data heterogeneity. However, system heterogeneity and device unreliability harm the FL process independently of data. For example, slower hardware may result in straggling clients which elongate rounds in synchronous FL~\citep{ScaleSystemDesign,FedProx} or operate on stale parameters in asynchronous FL~\citep{AsyncFedOpt,PAPAYA}\@. In addition, network or device unreliability creates dropout, which requires oversampling clients~\cite{ScaleSystemDesign} and harms the effectiveness of maintaining client state across rounds.

\paragraph{Dataset Shift and Continual Learning}  Allowing ML models to participate in lifelong learning effectively is the goal of continual learning~\citep{ContinualLearningSurvey}; however, applying continual learning to the FL context is problematic for two primary reasons. First, the optimisation objective (\cref{eq:flObjective}) intends to find a compromise model across all clients and cannot precisely fit all their data. Consequently, if the dataset of one client shifts independently of the whole network, the federated model will find it hard to adapt. Second, continual learning techniques such as Elastic-weight Consolidation~\citep{kirkpatrick2017overcoming},  PackNet~\citep{PackNetAM}, and Learning without Forgetting~\citep{LearningWithoutForgetting} are designed for task-incremental settings where class labels are known, small amounts of previous data may still be available for specialised use cases~\citep{kirkpatrick2017overcoming}, and there may even be different output heads for each task. The privacy requirements of FL make such solutions difficult at the level of the federated network without the addition of persistent local storage.
\section{Privacy}
While privacy in FL does not represent a main research direction for the proposed research, it is one of the primary concerns of the field. As such, any approaches which attempt to tackle the main challenges of FL must do so while accounting for their privacy implications.

Since FL keeps training data locally stored on the client, it offers more privacy than standard ML approaches. However, previous works~\citep{InvertingGradients,ProtectionAgainstReconstructionAndItsApplicationsInPrivateFL,PrivacyPreservingDLAdditivelyHomo,DeepLeakage} have shown that models trained in a federated fashion may allow for partial or complete reconstruction of their training data. From the perspective of the proposed research, privacy serves as a test for the feasibility of a particular method to be applied in an FL context. For example, methods which require detailed knowledge of the data each client~\citep{FLwithNonIID, OptimalUserEdgeAssingmentHierFL,FedHOME,CommEffDistillation} may be rejected as impractical. Similarly, the ability of a system to support FL privacy techniques like Secure Aggregation~(SecAgg) ~\citep{SecAggOG,FastSecAgg,LightSecAgg} or Differential Privacy~(DP)~\citep{DiffPrivacyOriginal,DiffPrivacyFL,LearningDifferentiallyPrivateRNNs,TowardsFairPrivacyPreservingFL,PracticalPrivateFLkairouz21b} is relevant.


Secure aggregation is a form of Secure Multi-party Aggregation and was introduced to FL by \citet{SecAggOG}. It operates by having all clients generate and share secrets. The clients then mask their models using random noise in such a manner that the server can construct the true average of client updates without knowing the true weights of any individual client. Such techniques require multiple clients to participate in aggregation concurrently and have $\mathcal{O}(C^2)$ communication cost where $C$ is the number of clients whose models are being aggregated. This excludes, for example, fully asynchronous federated learning as proposed by \citet{AsyncFedOpt}.

Differential Privacy in FL~\citep{LearningDifferentiallyPrivateRNNs} is formally defined as shown in \cref{eq:DiffPrivacy}

\begin{equation}\label{eq:DiffPrivacy}
    \mathrm{Pr}[M(d) \in S] \leq e^\epsilon \mathrm{Pr}[M(d^\prime) \in S] + \delta \ ,
\end{equation}
where $M$ is a probabilistic model, $S$ is the output set of that model, $d$ is the dataset used to train the model and $d^\prime$ is an adjacent dataset. Two datasets are adjacent in FL if they can be formed by adding or subtracting the local dataset of one client.  Finally, $(\epsilon,\delta)$ bound the similarity of outputs between two models trained with or without a specific client. Since DP comes with an inherent privacy-accuracy trade-off, the most relevant factor for its usability is whether an FL system can be scaled to operate over sufficiently large populations of clients. Larger populations allow productive training while offering a low $\epsilon$ by limiting the contribution of individual clients.

\section{Federated Learning Efficiency}\label{sec:back:FL_Efficiency}
It is now worth expanding on the trends that \citet{LargeCohorts} discovered. Those that limit the efficiency of FL in Non-IID settings where clients perform multiple SGD steps are of particular interest. Three significant effects can be observed. First, highly heterogeneous clients may cause sudden reductions in accuracy when their models are aggregated. Second, larger cohorts bring diminishing improvements in final accuracy and speed of convergence. Third, larger cohorts decrease data efficiency as more examples are needed for every accuracy gain.  More recently, \citet{UnderstandingModelAveragingInFL} have shown that while the federated model successfully keeps the client models in a common basin and achieves the lowest loss, it can drift from the optimum across rounds due to cohort heterogeneity. Furthermore, they also show that the efficacy of large cohorts in reducing the variance of aggregate updates is limited by the degree of data heterogeneity present.

These behaviours are approximately analogous to the well-known efficiency and generalisation limitations of large-batch training in centralised ML~\citep{LargeBatchGenGapSharpMinima}. \citet{LargeCohorts} find that data efficiency issues are caused by decreasing pseudo-gradient norms with increased cohort sizes and by the near-orthogonality of client updates following multiple steps of local training. The authors also find that adaptive optimisers fare better as cohort sizes grow due to scale invariance, making them particularly attractive aggregation algorithms.

\subsection{Adaptive Federated Optimisation}

Of particular relevance to the proposed research are Federated Averaging with Server Momentum (FedAvgM)~\citep{FedAvgM} and the more general Federated Adaptive Optimisation (FedOPT)~\citep{FedOPT}. They extend the concepts of momentum and adaptive optimisation~\citep{AdaGrad,Adam,SgdAlgoOverview} to Federated Learning on the \textit{server-side} by treating client updates as pseudo-gradients and maintaining information across rounds on server-side accumulators. This structure allows such strategies to minimise the impact of individual rounds by averaging their pseudo-gradients and derived quantities with those of previous rounds. Since the outcome of individual rounds is highly variable based on the combination of clients selected, such techniques offer a more consistent optimisation trajectory.

Specifically, following the account provided by \citet{FedOPT} as shown in \cref{eq:FedOpt:all}
\begin{subequations}
    \begin{align}
        \Delta_t     & = \cfrac{1}{|C|} \sum_{c \in C} \left( \theta_t^c - \theta_t \right) \label{eq:FedOpt:line-2} \\
        m_t          & = \beta_1 m_{t-1} + (1-\beta_1) \Delta_t \label{eq:FedOpt:line-3}                             \\
        v_t          & = \beta_2 v_t + (1-\beta_2) \Delta_t^2 \label{eq:FedOpt:line-4}                               \\
        \theta_{t+1} & = \theta_t + \eta \cfrac{m_t}{\sqrt{v_t} + \tau} \label{eq:FedOpt:line-5}
    \end{align}
    \label{eq:FedOpt:all}
\end{subequations}
for a given round $t$ and federated model $\theta_t$  each client $c$ in the selected set $C$ trains the model locally to construct a personalised version $\theta_t^c$. The pseudo-gradient $\Delta_t$ is then computed by averaging the differences between these personalised and federated models as shown in \cref{eq:FedOpt:line-2}. All operations on tensors are element-wise including division between tensors.

The first-moment accumulator $m_t$ can then be constructed as the weighted average of the previous accumulator $m_t$ and $\Delta_t$ using weight $\beta_1$ as shown in \cref{eq:FedOpt:line-3}. Thus, the pseudo-gradient of the current round is smoothed by those of the previous rounds decayed using $\beta_1$. Similarly, for the version of FedOpt based on Adam~\citep{Adam} the second-moment accumulator $v_t$  keeps track of the second power of the pseudo-gradient denoted by $\Delta_t^2$ as shown in \cref{eq:FedOpt:line-4}. These two accumulators are then used to compute the updated model for the next round $\theta_{t+1}$ using the server learning rate $\eta$ as shown in \cref{eq:FedOpt:line-5}. The term $\sqrt{v_t}$ normalises model parameters, making the algorithm scale-invariant to the pseudo-gradient. Finally, $\tau$ controls adaptivity\@.

FedOPT presents several promising properties in the context of hierarchical FL\@. First, \citet{FedOPT} show it is highly resilient to the exact choice of hyperparameters, including learning rate, compared to standard FedAvg and FedAvgM. Second, their scale-invariance partially addresses the issues observed by \citet{LargeCohorts} regarding the near-zero pseudo-gradients caused by the near-orthogonality of client updates. Third, they provide a means of automatically differentiating multiple servers based on accumulator state without hyperparameter tuning.

\section{Asynchronous Federated Learning}
Together with the previously mentioned adaptive federated optimization, asynchronous FL~\citep{AsyncFedOpt,FedBuff,PAPAYA,AsynchronousFLonHetDevicesSurvey,AsyncrhonousOnlineFL} represents another promising means of improving the overall efficiency of FL generally and the presently proposed system specifically by improving concurrency. In the context of FL, concurrency refers to \emph{the number of clients training simultaneously}.

In standard synchronous FL, the federated cohort size trained during a round controls the system's concurrency. Besides the data-efficiency issues discussed in \cref{sec:back:FL_Efficiency}, this round-based design introduces two factors which limit effective concurrency. First, since the new model is not obtained until the updates of the entire cohort have been averaged, the concurrency of a round decreases as clients finish training. Second, stragglers with slow hardware or large datasets elongate a round making it take as long as the slowest client. The first issue is intrinsic to the design, while the second is usually solved through oversampling~\citep{ScaleSystemDesign} or a time cut-off.

Asynchronous FL was proposed by \citet{AsyncFedOpt} as an alternative means of tackling stragglers in FL besides the standard oversampling method introduced by \citet{ScaleSystemDesign}. In the fully asynchronous form that \citet{AsyncFedOpt} proposes, it functions by allowing each client to update the global federated model directly whenever they have finished training. Thus, it removes the averaging of client updates seen in \cref{eq:FedAvg} and \cref{eq:FedOpt:line-2} and allows the system to maintain near-constant concurrency without being slowed down by stragglers. However, this introduces a new issue since clients may return stale updates at round $t$ generated by training the federated model from round $\tau$. As such, \citet{AsyncFedOpt} and future works~\citep{FedBuff,PAPAYA} utilise a staleness function $s(t-\tau)$ in the aggregation to account for this effect as seen in \cref{eq:FedAsync}.
\begin{equation} \label{eq:FedAsync}
    \theta_{t+1} = \theta_t + \eta  \left( s(t-\tau)\, \theta_{\tau}^c \right) \ .
\end{equation}
The benefits of the fully asynchronous approach are countermanded by its sensitivity to client data heterogeneity and its inability to properly utilize cohort-based techniques such as Secure Aggregation~\citep{SecAggOG}. Up to a limit~\citep{LargeCohorts,FedBuff,PAPAYA}, using cohort-based aggregation in synchronous FL imposes a variance-reduction effect which allows for more accurate estimation of $\Delta_t$ from the client updates $\Delta_t^c$ thus limiting the impact of highly heterogeneous clients. \citet{AsyncFedOpt} must instead adopt the local regularised of FedProx~\citep{FedProx} in order to constrain model divergence.

To regain the variance-reduction benefits and cohort-based techniques of synchronous FL, \citet{FedBuff} introduce FedBuff (Buffered Asynchronous Aggregation). FedBuff brings a conceptually minor but practically crucial change to async FL by introducing a buffer of size $K$, which holds updates until it is filled, after which it triggers averages of the pseudo-gradients and updates the model as in \cref{eq:FedAvg} with the potential addition of a staleness function. Unlike the synchronous approaches, concurrency remains decoupled from $K$ with $K$ used to control only how often a new model version is created rather than the number of training clients. \citet{PAPAYA} build upon FedBuff in deployment at Meta and show that it can bring significant improvements in convergence time by updating the model more frequently for the same concurrency and not waiting for stragglers. \citet{PAPAYA} also show FedBuff results in a more uniform accuracy distribution over clients than oversampling since it incorporates updates from such stragglers rather than completely discarding their contribution.

Asynchronous FL can be combined with adaptive optimisation, as done by \citet{PAPAYA}, communication-efficient methods like dropout regularisation~\citep{AsyncDropout}, and is entirely compatible with the hierarchical FL, as will be shown in \cref{sec:back:HFL} and \cref{sec:proposal}.
\section{Related work}\label{sec:back:related_work}

To tackle the inherent trade-off between optimising for the average global performance versus the performance on the data of a specific client which can be seen in \cref{eq:flObjective}, two overall directions emerged in the literature. The first, exemplified by Fair Federated Learning~\citep{QFedAvg}, attempts to modify the importance of a client in the federated objective function to change the final model's effectiveness for that client. The second relaxes the single global model requirement by personalising the federated model~\citep{SalvagingFL,TowardsPersonalisedFL,FLwithNonIID}, maintaining persistent fully-local models alongside it~\citep{Ditto}, clustering clients based on similarity~\citep{ThreeApproachesMansour,AnEfficientFrameworkForClusteredFL}, or building hierarchies~\citep{Client-Edge-CloudHierFL,Hier_Het_Cellular}. Since the proposed B-HFL family of algorithms falls in the second camp, this section shall detail the most closely related work and present its limitations. Finally, the desired properties of the federated system and their relation to previous work are summarised in \cref{tab:gap_analysis}.


\subsection{Personalised Federated Learning}

Personalised Federated Learning (PFL) refers to a class of FL algorithms which intends to tackle the Non-IID distribution of client data by creating models which better match the distribution of specific clients or groups of clients. This differs from standard FL where the goal is the creation of the best compromise model. Thus, PFL approaches lie on a spectrum between creating one global model and personalising on a per-client basis. For the proposed research, of concern are per-client personalisation and clustering.

\subsubsection{Per-client personalisation}

Fully personalised FL refers to creating one model per client in addition to the global one. The most common means of achieving this is by local adaptation, or fine-tuning, of the federated model after training~\citep{SalvagingFL}. This is potentially combined with techniques such as Knowledge Distillation~\citep{DeepMutualLearning} or Elastic-weight Consolidation~\citep{kirkpatrick2017overcoming} for the explicit purpose of combating catastrophic forgetting~\citep{CatForgetting1}. However, this two-stage optimisation is challenging to implement in an FL lifecycle where the federated model may need additional training after the adaptation phase has already been carried out. Furthermore, it provides no middle ground between the global and local models, which hurts the ability of such systems to integrate new clients, which may be incapable of fine-tuning.


Multi-task learning represenst another promising avenue for creating personalised models by treating each client as a task. Common approaches such as those of \todo{add approahces} operate by learining the pairwise relationships between clients.

A more recent approach is represented by Ditto~\citep{Ditto} for settings where clients are visited frequently and can maintain state across rounds. Ditto allows clients to maintain a persistent local model and train it alongside the federated one during FL rounds. The two models are connected by incorporating the $l_2$ distance between their weights within the loss function of the local one.  Model interpolation~\citep{ThreeApproachesMansour,AdaptivePersonalisedFederatedLearning,FederatedLearningMixtureOfGlobalAndLocal} and techniques which maintain local personalisation layers~\citep{FlWithPersonalisationLayers,FedSplitBert} also rely on clients being capable of maintaining a persistent state. The model interpolation approaches of \citet{ThreeApproachesMansour} and \citet{AdaptivePersonalisedFederatedLearning} rely on optimising local models with a mixture parameter which is adaptively tuned through SGD. On the other hand, Looples Gradient Descent~\citep{FederatedLearningMixtureOfGlobalAndLocal} allows clients to probabilistically take steps towards either local training or partially averaging the federated model into their local one. FInally, split approaches such as those of \citet{FlWithPersonalisationLayers} and \citet{FedSplitBert} train models locally but only average and update the layers up to a cut layer $c$, with the intuition that earlier layers represent more general feature extractors and later layers require heavier personalisation.

However, despite proven benefits to personalisation~\citep{FlWithPersonalisationLayers,AdaptivePersonalisedFederatedLearning} as well as fairness and robustness~\citep{Ditto}, persistent local models still face the challenges of traditional personalised models. Moreover, they do not address dataset shifts within the client, as they only operate during training rounds.

\subsubsection{Clustering}

Clustering clients is a technique that attempts to group participants based on a similarity metric. These clusters are constructed using two primary approaches. First, since directly clustering clients based on their data is unfeasible in FL, standard clustering approaches such as K-means~\citep{K-means} or Hierarchical Agglomerative Clustering~\citep{OgHierClustering} need to operate over embeddings which represent the local distribution of a client. The natural choice for such an embedding in FL is using locally-trained models (or pseudo-gradients) directly, as done in the one-shot K-means algorithm of \citet{AnEfficientFrameworkForClusteredFL} and the hierarchical clustering algorithm of \citet{HierClustering}. In such solutions, the distance function between models will be either the $l_n$ norm of model differences or a similarity metric such as the cosine similarity computed over flattened parameters. While locally trained models are already widely available and more admissible from a privacy perspective to a more raw representation like prototypes~\citep{FedProto}, using the entire model is both computationally expensive and potentially uninformative. As \citet{FedMA} indicates, two models may represent the same concept in different sets of weights due to the permutation invariance of the hidden layers. This issue can be addressed using the final layer to cluster either the target model or a dedicated encoder~\citep{CommunityBasedFL}.

Second, because the methods above are computationally expensive and challenging to apply dynamically for low-participation cross-device FL, \citet{AnEfficientFrameworkForClusteredFL} and \citet{ThreeApproachesMansour} concomitantly proposed a loss-based form of clustering for FL. Their algorithms are iterative and operate by maintaining $K$ hypothesis models, which are communicated to clients. Clients then get assigned to the cluster model where they have the lowest training loss, and then the models of clients who are self-selected to a specific cluster are averaged to produce the new cluster models. The procedure then repeats for several rounds or until convergence; at this point, each cluster is separated and runs FL independently with their assigned clients. Clusters may also exist naturally based on characteristics like geographic location or language.





\subsection{Hierarchical Federated Learning}\label{sec:back:HFL}


The most relevant subfield of FL to our proposal is Hierarchical Federated Learning (HFL) introduced by \citet{Client-Edge-CloudHierFL}. Their proposed HierFAVG algorithm was developed primarily to handle the communication challenges of traditional cloud-based FL\@. In order to obtain scales of millions of participating clients~\citep{GoogleKeyboard, ScaleSystemDesign}, FL systems relied on cloud infrastructure to connect devices over a wide geographic area and thus incurred additional latency. This trade-off was considered worthwhile since the larger populations were necessary for convergence, and edge servers, while capable of fast client communication, could not draw on a sufficient data pool. \citet{Client-Edge-CloudHierFL} argue that a two-level structure resolves the tensions between edge servers close to the clients and cloud servers. \citet{Hier_Het_Cellular} propose an identical algorithm for heterogeneous celluar networks where edge servers are small cell base stations, and a central macro base station replaces the cloud server. Similarly to \citet{Client-Edge-CloudHierFL}, \citet{Hier_Het_Cellular} focus on reducing communication costs and go further in this direction by utilising update sparsification techniques~\citep{DeepGradientCompressin,CommCompressionDecent}. To further improve communication efficiency \citet{HFELJointEdgeResource} propose a network and compute-aware resource allocation framework for hierarchical FL, which assigns clients to edge servers to optimise costs.


Previous works in HFL show a series of limitations. The HierFAVG algorithm directly extends FedAvg~\citep{FedAvg} by allowing the cloud server to treat edge servers as clients. However, because \citet{Client-Edge-CloudHierFL} and \citet{Hier_Het_Cellular} only consider communication efficiency, they do not allow the edge servers to maintain greater personalisation and instead replace their model entirely during cloud-aggregation. Furthermore, their system does not consider asynchronicity, proxy training, or multi-level hierarchies. Regarding clustering, the available algorithms fail to obtain the desired trade-off between generalisation and personation. Standard clustering algorithms in FL assume data-sharing between clusters is unnecessary and do not directly map onto a hierarchical communication structure. One execption to this is the weight-sharing iterative clustering algorithm proposed by \citet{AnEfficientFrameworkForClusteredFL} which serves as a middle point between a two-layer hierarchical solution and multi-task learning with personalisation layers. Finally, clustering algorithms are not generally meant to provide a single global model besides cluster models for applications where it would be beneficial.


The work of \citet{ResourceEfficientHierAgg} combines hierarchical aggregation and clustering in a mixed scenario of peer-to-peer and client-server FL where powerful clients take on the role of edge servers and perform aggregation before transmitting their models to the cloud. However, their clustering procedure is meant to optimise communication efficiency first and foremost while satisfying arbitrary resource constraints. It thus does not exploit the personalisation advantages of combining clustering and hierarchical FL\@.

\citet{OptimalUserEdgeAssingmentHierFL} do consider scenarios where the data distribution of edge servers is taken into account. Specifically they allow edge servers to contain clients with a Non-IID distribution and use FedSGD~\citep{FedAvg} to counteract its effects. To obtain communication efficiency without sacrificing convergence at the cloud server, they attempt to maintain an IID distribution across edge servers and apply FedAvg at the cloud server level. While promising, their work requires complete knowledge of the distribution of each client in order to realise edge-server assignment. Furthermore, it assumes that edge servers have sufficiently low communication latency to efficiently train with FedSGD despite the original work of \citet{FedAvg} showing FedSGD to be up to two orders of magnitude slower than FedAvg in terms of convergence speed.




% Please add the following required packages to your document preamble:
% \usepackage{booktabs}
% \usepackage{graphicx}
% \usepackage[normalem]{ulem}
% \useunder{\uline}{\ul}{}
\begin{table}[h]
    \centering
    \caption{Gap analysis table showing proposed system's properties and overlap with closely related work.}
    \resizebox{\columnwidth}{!}{%
        \begin{tabular}{@{}lcccccc@{}}
            \toprule
            Related Work                               & Hierarchical Structure          & Personalisation                 & Allows Persistent Models        & General Group Models            & Meaningful Group Models         & Asynchronous Work               \\ \midrule
            Local Adaptation                           &                                 & {\color{ForestGreen}\checkmark} &                                 &                                 &                                 &                                 \\
            Ditto, Model Interpolation, Split Learning &                                 & {\color{ForestGreen}\checkmark} & {\color{ForestGreen}\checkmark} &                                 &                                 &                                 \\
            Clustering                                 &                                 &                                 &                                 &                                 &                                 &                                 \\
            HieFAVG                                    & {\color{ForestGreen}\checkmark} &                                 &                                 & {\color{ForestGreen}\checkmark} & {\color{ForestGreen}\checkmark} &                                 \\
            Asynchronous FL                            &                                 &                                 &                                 &                                 &                                 & {\color{ForestGreen}\checkmark} \\
            Bidirectional Hierarchical FL              & {\color{ForestGreen}\checkmark} & {\color{ForestGreen}\checkmark} & {\color{ForestGreen}\checkmark} & {\color{ForestGreen}\checkmark} & {\color{ForestGreen}\checkmark} & {\color{ForestGreen}\checkmark} \\ \bottomrule
        \end{tabular}%
    }\label{tab:gap_analysis}
\end{table}
