Propunerea din acest document a apărut ca o consecință naturală a cercetărilor privind Învățarea Federată Personalizată și Învățarea Federată Ierarhică pe care le-am început în timpul MPhil-ului meu în Știința Calculatoarelor Avansată și în primul an al doctoratului meu în laboratorul Cambridge ML Systems, condus de supervisorul meu, Dr. Nicholas Lane.

\citet{EuroMLSysWorkshop} a investigat compromisul dintre generalizare și personalizare, care este în centrul acestui studiu, din perspectiva Învățării Federate Corecte și a interacțiunilor sale cu adaptarea locală~(fine-tuning) a modelului federat post-antrenament. Deoarece Învățarea Federată Corectă încearcă să construiască o distribuție mai uniformă a acurateții pentru modelul federat pe seturile de test locale ale clienților, așteptarea era fie să reducă necesitatea personalizării, fie să ofere un punct de plecare mai bun din care să o efectueze. Rezultatele experimentale au arătat că FL corectă nu aduce beneficii și potențiale dezavantaje în direcția personalizării ulterioare și a dus la propunerea unui algoritm FL conștient de personalizare care încearcă să anticipeze reglementările comune utilizate în timpul fine-tuning-ului pe parcursul procesului FL.

\citet{OperaWorkshop} a evaluat performanța Recunoașterii Activităților Umane Federate~\citep{HARusingFL_2018} când a fost antrenată folosind date multimodale adunate de la diferite tipuri de senzori la niveluri crescute de confidențialitate. A demonstrat că gruparea clienților în funcție de tipul de senzor care a produs setul lor de antrenament a atenuat eficient impactul necesității de confidențialitate la nivel de subiect uman, mediu și senzor simultan. A fost un precursor direct al Învățării Federate Ierarhice Bidirecționale, deoarece se baza pe o structură de model în două niveluri în care fiecare client antrena atât un model la nivel de grup, cât și modelul federat global, folosind o abordare de învățare mutuală~\citep{DeepMutualLearning}. Această lucrare a fost ulterior extinsă pentru a lua în considerare adaptabilitatea unor astfel de sisteme în două niveluri la adăugarea unui nou tip de senzor~(grup) în federat; extinderea a fost trimisă simpozionului \href{https://mobiuk.org/2023}{MobiUK}. Învățarea mutuală a fost aleasă pentru a relaționa modelele la nivel de grup și global deoarece a permis arhitecturi divergente care împărțeau doar stratul de ieșire.