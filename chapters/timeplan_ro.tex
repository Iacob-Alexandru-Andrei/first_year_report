Familia de algoritmi prezentată pentru Învățarea Federată Ierarhică Bidirecțională va fi dezvoltată în perioada doctoratului și va face parte din teza finală de doctorat. În plus, înainte de teza finală, oferă oportunități pentru publicații la conferințe ce contribuie semnificativ la FL. Având în vedere noutatea FL în general și a HFL în particular, există un spațiu larg pentru dezvoltări ulterioare în structura B-HFL pe măsură ce domeniile se maturizează. Perioada de vară de la sfârșitul primului meu an de doctorat va fi dedicată implementării versiunii exemplu a B-HFL în cadrul framework-ului FL Flower~\citep{Flower} afiliat grupului nostru de cercetare. Acest framework este în prezent reglat pentru setările standard FL și ar necesita modificări importante ale API-ului pentru a executa și simula eficient HFL\@. Lucrările anterioare privind modelele la nivel de grup pentru Recunoașterea Activității Umane Federate ale \citet{OperaWorkshop} și motorul eficient de simulare FL la care am contribuit pot fi baza pentru implementarea sistemului.

Semestrul de toamnă, Michaelmas, al celui de-al doilea an va avea ca obiectiv principal publicarea unui articol de conferință bazat pe sistemul exemplu propus în Secțiunea \ref{sec:example_system}. Am discutat deja acest lucru cu supervizorul meu, Dr. Nicholas Lane, și am convenit că atât \href{https://iclr.cc/}{ICLR} cât și \href{https://mlsys.org/}{MLSys} ar fi conferințe potrivite. Având în vedere importanța crescândă a LLM-urilor și compromisurile recent descoperite de \citet{PersonalisationGeneralisationTradeoff} în ceea ce privește abilitățile lor de generalizare și personalizare, ele reprezintă o aplicație naturală pentru sistemul ierarhic propus. Mai mult, predicția textului în mai multe limbi oferă o aplicație FL, grupată în mod natural, corespunzătoare scenariilor din lumea reală în care țările au servere independente edge pentru FL și trebuie să colaboreze la un nivel continental și global. Studiul ar folosi un model BERT multilingv mare~\citep{RoBERTA} împreună cu două seturi de date multilingve~\citep[e.g., ][]{XGLUE,mC4} pentru antrenament. Un set de date va fi împărțit după limbă, iar celălalt va fi păstrat ca un set de date proxy la serverul central din \cref{fig:TreeStructure}. Obiectivele studiului ar fi să compare acuratețea finală a fiecărui model la fiecare nivel al ierarhiei pe seturile de test ale clienților și setul de test centralizat partiționat din setul de date proxy inițial. Așteptarea ar fi ca performanța modelului pe datele unui anumit client să fie proporțională cu proximitatea acestuia față de acel client în arbore. Alternativ, pentru setul de test proxy și uniunea tuturor seturilor de test ale clienților, acuratețea ar trebui să fie proporțională cu proximitatea față de serverul central. În plus, studii de ablație privind conexiunile "reziduale" sau optimizarea adaptivă vor fi efectuate. În cele din urmă, dacă timpul permite, lucrarea poate include alte sarcini grupate natural, cum ar fi recunoașterea vorbirii.

După publicarea acestei lucrări, o extensie naturală în semestrele Lent și Easter ar fi abordarea unui mediu în care clienții generează și șterg continuu date având memorie limitată. Sistemul exemplu ar fi extins pentru a permite antrenament asincron pe toate nodurile, inclusiv frunzele, care rulează în paralel cu componenta FL reală. Fiecare client ar genera un flux de date având o memorie internă fixă pe care să opereze în timpul antrenamentului. Constrângerile reale de resurse și asincronia pot fi modelate folosind clusterul Raspberry Pi FL de la Cambridge ML Systems. Această lucrare ar fi destinată pentru \href{https://sigmobile.org/mobicom/2023/}{MobiCom}, același loc unde am prezentat motorul de simulare Flower.

Dacă propunerea este reușită, cel de-al doilea an al doctoratului meu ar aduce o contribuție valoroasă în domeniul FL și s-ar concretiza în una sau mai multe publicații la conferințe împreuna cu o parte din teza finală. De asemenea, ar reprezenta o extensie majoră a framework-ului Flower~\citep{Flower} cu potențial pentru viitoare colaborări sau angajări în startup-ul \href{https://flower.dev/blog/2023-03-08-flower-labs/}{Flower Labs} finanțat de \href{https://www.ycombinator.com/}{Y Combinator}. După finalizarea doctoratului, intenționez să urmez o carieră în cercetarea privată sau academică.